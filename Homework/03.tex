\documentclass[11pt]{article}
\usepackage[letterpaper]{geometry}
\usepackage{MATH561}
\usepackage{SetTheory}
\usepackage{Derivative}
\allowdisplaybreaks

\begin{document}
\noindent \textbf{\Large{Caleb Logemann \\
MATH 561 Numerical Analysis I \\
Homework 3
}}

\begin{enumerate}
    \item % 1
    \begin{enumerate}
        \item[(a)] Use the central quotient approximation $f'(x) \approx
            \frac{f(x + h) - f(x - h)}{2h}$ to obtain an approximation of
            $\frac{\partial^2}{\partial x \partial y} u(x,y)$, for a function
            $u$ of two variables.
            \begin{align*}
                \frac{\partial^2}{\partial x \partial y} u(x,y) &=
                    \frac{\partial}{\partial x} \p{\frac{\partial}{\partial y} u(x,y)} \\
                &\approx \frac{\partial}{\partial x} \p{\frac{u(x,y + h) - u(x,y - h)}{2h}} \\
                &= \frac{1}{2h} \p{\frac{\partial}{\partial x} u(x,y+h) -
                    \frac{\partial}{\partial x} u(x,y-h)} \\
                &\approx \frac{1}{2h} \p{\frac{u(x+h, y+h) - u(x-h, y+h)}{2h} -
                    \frac{u(x+h, y-h) - u(x-h, y-h)}{2h}} \\
                &= \frac{u(x+h, y+h) - u(x-h, y+h) - u(x+h, y-h) + u(x-h, y-h)}{4h^2} \\
            \end{align*}

        \item[(b)]
            The fourth order Taylor expansion of $u(v,w)$ approximated at $(x,y)$ is
            \begin{align*}
                u(v,w) &= u(x,y) + (v - x)\pd{u(x,y)}{x} + (w - y)\pd{u(x,y)}{y}
                + \frac{(v-x)^2}{2} \pd[2]{u(x,y)}{x} \\
                &+ (v-x)(w-y)\mpd[2]{u(x,y)}{\partial x \partial y}
                + \frac{(w-y)^2}{2}\pd[2]{u(x,y)}{y}
                + \frac{(v-x)^3}{3!}\pd[3]{u(x,y)}{x} \\
                &+ \frac{(v-x)^2(w-y)}{2!} \mpd[3]{u(x,y)}{\partial x^2 \partial y}
                + \frac{(v-x)(w-y)^2}{2!} \mpd[3]{u(x,y)}{\partial x \partial y^2} \\
                &+ \frac{(w-y)^3}{3!} \pd[3]{u(x,y)}{y}
                + \frac{(v-x)^4}{4!} \pd[4]{u(x,y)}{x}
                + \frac{(v-x)^3(w-y)}{3!} \mpd[4]{u(x,y)}{\partial x^3 \partial y} \\
                &+ \frac{(v-x)^2 (w-y)^2}{2! 2!} \mpd[4]{u(x,y)}{\partial x^2 \partial y^2}
                + \frac{(v-x) (w-y)^3}{3!} \mpd[4]{u(x,y)}{\partial x \partial y^3} \\
                &+ \frac{(w-y)^4}{4!} \pd[4]{u(x,y)}{y}
            \end{align*}

            Taking the fourth order Taylor expansion of the terms found in part (a)
            \begin{align*}
                &\frac{u(x+h, y+h) - u(x-h, y+h) - u(x+h, y-h) + u(x-h, y-h)}{4h^2} \\
                &\approx \frac{1}{4h^2} (u(x,y) + h\pd{u(x,y)}{x} + h\pd{u(x,y)}{y}
                    + \frac{h^2}{2} \pd[2]{u(x,y)}{x} \\
                    &+ h^2\mpd[2]{u(x,y)}{\partial x \partial y}
                    + \frac{h^2}{2}\pd[2]{u(x,y)}{y}
                    + \frac{h^3}{3!}\pd[3]{u(x,y)}{x} \\
                    &+ \frac{h^3}{2!} \mpd[3]{u(x,y)}{\partial x^2 \partial y}
                    + \frac{h^3}{2!} \mpd[3]{u(x,y)}{\partial x \partial y^2} \\
                    &+ \frac{h^3}{3!} \pd[3]{u(x,y)}{y}
                    + \frac{h^4}{4!} \pd[4]{u(x,y)}{x}
                    + \frac{h^4}{3!} \mpd[4]{u(x,y)}{\partial x^3 \partial y} \\
                    &+ \frac{h^4}{2! 2!} \mpd[4]{u(x,y)}{\partial x^2 \partial y^2}
                    + \frac{h^4}{3!} \mpd[4]{u(x,y)}{\partial x \partial y^3} \\
                    &+ \frac{h^4}{4!} \pd[4]{u(x,y)}{y} - 
                    u(x,y) + h\pd{u(x,y)}{x} - h\pd{u(x,y)}{y}
                    - \frac{h^2}{2} \pd[2]{u(x,y)}{x} \\
                    &+ h^2\mpd[2]{u(x,y)}{\partial x \partial y}
                    - \frac{h^2}{2}\pd[2]{u(x,y)}{y}
                    + \frac{h^3}{3!}\pd[3]{u(x,y)}{x} \\
                    &- \frac{h^3}{2!} \mpd[3]{u(x,y)}{\partial x^2 \partial y}
                    + \frac{h^3}{2!} \mpd[3]{u(x,y)}{\partial x \partial y^2} \\
                    &- \frac{h^3}{3!} \pd[3]{u(x,y)}{y}
                    - \frac{h^4}{4!} \pd[4]{u(x,y)}{x}
                    + \frac{h^4}{3!} \mpd[4]{u(x,y)}{\partial x^3 \partial y} \\
                    &- \frac{h^4}{2! 2!} \mpd[4]{u(x,y)}{\partial x^2 \partial y^2}
                    + \frac{h^4}{3!} \mpd[4]{u(x,y)}{\partial x \partial y^3} \\
                    &- \frac{h^4}{4!} \pd[4]{u(x,y)}{y} - u(x,y) - h\pd{u(x,y)}{x} + h\pd{u(x,y)}{y}
                    - \frac{h^2}{2} \pd[2]{u(x,y)}{x} \\
                    &+ h^2\mpd[2]{u(x,y)}{\partial x \partial y}
                    - \frac{h^2}{2}\pd[2]{u(x,y)}{y}
                    - \frac{h^3}{3!}\pd[3]{u(x,y)}{x} \\
                    &+ \frac{h^3}{2!} \mpd[3]{u(x,y)}{\partial x^2 \partial y}
                    - \frac{h^3}{2!} \mpd[3]{u(x,y)}{\partial x \partial y^2} \\
                    &+ \frac{h^3}{3!} \pd[3]{u(x,y)}{y}
                    - \frac{h^4}{4!} \pd[4]{u(x,y)}{x}
                    + \frac{h^4}{3!} \mpd[4]{u(x,y)}{\partial x^3 \partial y} \\
                    &- \frac{h^4}{2! 2!} \mpd[4]{u(x,y)}{\partial x^2 \partial y^2}
                    + \frac{h^4}{3!} \mpd[4]{u(x,y)}{\partial x \partial y^3} \\
                    &- \frac{h^4}{4!} \pd[4]{u(x,y)}{y}
                    + u(x,y) - h\pd{u(x,y)}{x} - h\pd{u(x,y)}{y}
                    + \frac{h^2}{2} \pd[2]{u(x,y)}{x} \\
                    &+ h^2\mpd[2]{u(x,y)}{\partial x \partial y}
                    + \frac{h^2}{2}\pd[2]{u(x,y)}{y}
                    - \frac{h^3}{3!}\pd[3]{u(x,y)}{x} \\
                    &- \frac{h^3}{2!} \mpd[3]{u(x,y)}{\partial x^2 \partial y}
                    - \frac{h^3}{2!} \mpd[3]{u(x,y)}{\partial x \partial y^2} \\
                    &- \frac{h^3}{3!} \pd[3]{u(x,y)}{y}
                    + \frac{h^4}{4!} \pd[4]{u(x,y)}{x}
                    + \frac{h^4}{3!} \mpd[4]{u(x,y)}{\partial x^3 \partial y} \\
                    &+ \frac{h^4}{2! 2!} \mpd[4]{u(x,y)}{\partial x^2 \partial y^2}
                    + \frac{h^4}{3!} \mpd[4]{u(x,y)}{\partial x \partial y^3} \\
                    &+ \frac{h^4}{4!} \pd[4]{u(x,y)}{y}) \\
%                &= \frac{1}{4h^2} (
%%                    + \frac{h^2}{2} \pd[2]{u(x,y)}{x} \\
%                    &+ h^2\mpd[2]{u(x,y)}{\partial x \partial y}
%                    + \frac{h^2}{2}\pd[2]{u(x,y)}{y}
%                    + \frac{h^3}{3!}\pd[3]{u(x,y)}{x} \\
%                    &+ \frac{h^3}{2!} \mpd[3]{u(x,y)}{\partial x^2 \partial y}
%                    + \frac{h^3}{2!} \mpd[3]{u(x,y)}{\partial x \partial y^2} \\
%                    &+ \frac{h^3}{3!} \pd[3]{u(x,y)}{y}
%                    + \frac{h^4}{4!} \pd[4]{u(x,y)}{x}
%                    + \frac{h^4}{3!} \mpd[4]{u(x,y)}{\partial x^3 \partial y} \\
%                    &+ \frac{h^4}{2! 2!} \mpd[4]{u(x,y)}{\partial x^2 \partial y^2}
%                    + \frac{h^4}{3!} \mpd[4]{u(x,y)}{\partial x \partial y^3} \\
%                    &+ \frac{h^4}{4!} \pd[4]{u(x,y)}{y}
%                    - \frac{h^2}{2} \pd[2]{u(x,y)}{x} \\
%                    &+ h^2\mpd[2]{u(x,y)}{\partial x \partial y}
%                    - \frac{h^2}{2}\pd[2]{u(x,y)}{y}
%                    + \frac{h^3}{3!}\pd[3]{u(x,y)}{x} \\
%                    &- \frac{h^3}{2!} \mpd[3]{u(x,y)}{\partial x^2 \partial y}
%                    + \frac{h^3}{2!} \mpd[3]{u(x,y)}{\partial x \partial y^2} \\
%                    &- \frac{h^3}{3!} \pd[3]{u(x,y)}{y}
%                    - \frac{h^4}{4!} \pd[4]{u(x,y)}{x}
%                    + \frac{h^4}{3!} \mpd[4]{u(x,y)}{\partial x^3 \partial y} \\
%                    &- \frac{h^4}{2! 2!} \mpd[4]{u(x,y)}{\partial x^2 \partial y^2}
%                    + \frac{h^4}{3!} \mpd[4]{u(x,y)}{\partial x \partial y^3} \\
%                    &- \frac{h^4}{4!} \pd[4]{u(x,y)}{y}
%                    - \frac{h^2}{2} \pd[2]{u(x,y)}{x} \\
%                    &+ h^2\mpd[2]{u(x,y)}{\partial x \partial y}
%                    - \frac{h^2}{2}\pd[2]{u(x,y)}{y}
%                    - \frac{h^3}{3!}\pd[3]{u(x,y)}{x} \\
%                    &+ \frac{h^3}{2!} \mpd[3]{u(x,y)}{\partial x^2 \partial y}
%                    - \frac{h^3}{2!} \mpd[3]{u(x,y)}{\partial x \partial y^2} \\
%                    &+ \frac{h^3}{3!} \pd[3]{u(x,y)}{y}
%                    - \frac{h^4}{4!} \pd[4]{u(x,y)}{x}
%                    + \frac{h^4}{3!} \mpd[4]{u(x,y)}{\partial x^3 \partial y} \\
%                    &- \frac{h^4}{2! 2!} \mpd[4]{u(x,y)}{\partial x^2 \partial y^2}
%                    + \frac{h^4}{3!} \mpd[4]{u(x,y)}{\partial x \partial y^3} \\
%                    &- \frac{h^4}{4!} \pd[4]{u(x,y)}{y}
%                    + \frac{h^2}{2} \pd[2]{u(x,y)}{x} \\
%                    &+ h^2\mpd[2]{u(x,y)}{\partial x \partial y}
%                    + \frac{h^2}{2}\pd[2]{u(x,y)}{y}
%                    - \frac{h^3}{3!}\pd[3]{u(x,y)}{x} \\
%                    &- \frac{h^3}{2!} \mpd[3]{u(x,y)}{\partial x^2 \partial y}
%                    - \frac{h^3}{2!} \mpd[3]{u(x,y)}{\partial x \partial y^2} \\
%                    &- \frac{h^3}{3!} \pd[3]{u(x,y)}{y}
%                    + \frac{h^4}{4!} \pd[4]{u(x,y)}{x}
%                    + \frac{h^4}{3!} \mpd[4]{u(x,y)}{\partial x^3 \partial y} \\
%                    &+ \frac{h^4}{2! 2!} \mpd[4]{u(x,y)}{\partial x^2 \partial y^2}
%                    + \frac{h^4}{3!} \mpd[4]{u(x,y)}{\partial x \partial y^3} \\
%                    &+ \frac{h^4}{4!} \pd[4]{u(x,y)}{y}) \\
                &= \frac{1}{4h^2} \p{\frac{2h^4}{3}
                    \mpd[4]{u(x,y)}{\partial x^3 \partial y} + 
                    4h^2 \mpd[2]{u(x,y)}{\partial x \partial y} + \frac{2h^4}{3}
                    \mpd[4]{u(x,y)}{\partial x \partial y^3}} \\
                &= \mpd[2]{u(x,y)}{\partial x \partial y} + \frac{h^2}{6} 
                    \p{\mpd[4]{u(x,y)}{\partial x^3 \partial y} + 
                    \mpd[4]{u(x,y)}{\partial x \partial y^3}}\\
                &= \mpd[2]{u(x,y)}{\partial x \partial y} + O(h^2)
            \end{align*}
    \end{enumerate}

    % 2 TODO: finish part (b)
    \item Let $s$ be a function defined by
        \begin{align*}
            s(x) &=
            \begin{cases}
                (x + 1)^3 & -1 \le x \le 0 \\
                (1 - x)^3 & 0 \le x \le 1
            \end{cases}
        \end{align*}
    \begin{enumerate}
        \item[(a)]
            With $\Delta$ denoting the subdivision of $\br{-1,1}$ into
            $\br{-1, 0}$ and $\br{0, 1}$, to what class $S_m^k(\Delta)$
            does the spline $s$ belong to?

            Since each piece of $s$ is degree 3, the degree of $s$ is $m = 3$.
            Let $s_1(x) = (x + 1)^3$ and let $s_2(x) = (1-x)^3$.
            Then $s$ is continuous because $s_1(0) = (0 + 1)^3 = 1 = (1 - 0)^3
            = s_2(0)$.
            Also $s_1'(0) = 3(0 + 1)^2 = 3$ and $s_2'(0) = -3(1 - 0)^2 = -3$,
            therefore the first derivative of $s$ is not continuous.
            So $s$ belongs to smoothness class $k=0$.

        \item[(b)] % TODO: find error estimate of trapezoidal rule
            Estimate the error of the composite trapezoidal rule applied to 
            $\dintt{-1}{1}{s(x)}{x}$, when $\br{-1, 1}$ is divided into n
            subintervals of equal length $h = 2/n$ and n is even.

            

        \item[(c)]
            What is the error of the composite Simpson’s rule applied to
            $\dintt{-1}{1}{s(x)}{x}$, with the same subdivision of
            $\br{-1, 1}$ as in (b)?

            Simpson's rule has a degree of exactness equal to 3.
            Simpson's rule is applied to every two intervals, since $n$ is
            even Simpson's rule can be applied to $s$ over the subdivision
            $\Delta$.

            Since $n$ is even either $n = 4m$ or $n = 4m + 2$ for some positive
            integer $m$.
            If $n = 4m$ for some positive integer $m$, that is $n$ is a multiple
            of $4$, then $\dintt{-1}{1}{s(x)}{x}$ can be approximated by
            applying Simpson's rule to $\dintt{-1}{0}{s(x)}{x}$ and 
            $\dintt{0}{1}{s(x)}{x}$ separately and summing.
            This can be done because there $n/2 = 2m$ intervals on $\br{-1, 0}$ and
            $\br{0,1}$.
            Each of these integrals can be evaluated exactly because Simpson's rule
            has degree of exactness equal to 3.
            Therefore the total error is 0.

            If $n = 4m + 2$ for some positive integer $m$, then 
            $\dintt{-1}{1}{s(x)}{x}$ can be approximated by
            applying Simpson's rule to $\dintt{-1}{-h}{s(x)}{x}$, 
            $\dintt{-h}{h}{s(x)}{x}$, and $\dintt{h}{1}{s(x)}{x}$ separately
            and summing.
            In this situation each interval $\br{-1,0}$ and $\br{0,1}$ has
            an odd number of subintervals, so Simpson's rule must be applied
            across the interval $\br{-h, h}$.
            Simpson's rule evaluates $\dintt{-1}{-h}{s(x)}{x}$ and
            $\dintt{h}{1}{s(x)}{x}$ exactly because $s(x)$ is a degree 3
            polynomial on these intervals.
            Therefore the error from Simpson's rule comes when approximating
            the integral $\dintt{-h}{h}{s(x)}{x}$.
            The error can be found as follows.
            \begin{align*}
                E &= \dintt{-h}{h}{s(x)}{x} - \frac{h}{3} \p{s(-h) + 4s(0) + s(h)} \\
                  &= \dintt{-h}{0}{(x+1)^3}{x} + \dintt{0}{h}{(1-x)^3}{x} - \frac{h}{3} \p{(1-h)^3 + 4 + (1-h)^3} \\
                  &= \eval{\frac{1}{4}(x+1)^4}{x=-h}{0} + \eval{-\frac{1}{4}(1-x)^4}{x=0}{h} - \frac{h}{3} \p{2(1-h)^3 + 4} \\
                  &= \frac{1}{4}\p{1 - (1-h)^4} - \frac{1}{4}\p{(1-h)^4 - 1} - \frac{h}{3} \p{2(1-h)^3 + 4} \\
                  &= \frac{1}{2} - \frac{1}{2}(1-h)^4 - \frac{4h}{3} - \frac{2h}{3}(1-h)^3 \\
                  &= (1-h)^3\p{-\frac{1}{2}(1-h) - \frac{2h}{3}} - \frac{4h}{3} + \frac{1}{2} \\
                  &= (1-h)^3\p{-\frac{h}{6} -\frac{1}{2}} - \frac{4h}{3} + \frac{1}{2} \\
            \end{align*}
            This is also the total error.

        \item[(d)]
            What is the error resulting from applying the 2-point
            Gauss-Legendre rule to $\dintt{-1}{0}{s(x)}{x}$ and
            $\dintt{0}{1}{s(x)}{x}$ separately and summing?

            The 2-point Gauss-Legendre rule has degree of exactness equal to 3.
            So on each of these intervals the $s(x)$ is a degree 3 polynomial,
            therefore the error on each of these intervals will be zero.
            So the total error is zero.
    \end{enumerate}

    \item % 3
    \begin{enumerate}
        \item[(a)] %TODO: look at error of intrpolation polynomial
            Determine by Hermite interplation the quadractic polynomial $p$
            interpolating $f$ at $x=0$ and $x=1$ and $f'$ at $x = 0$.
            Also express the errors in terms of an appropriate derivative.

            \begin{tabular}{c|c|c|c}
                $x$ & $f(x)$ & & \\
                \midrule
                $0$ & $f(0)$ & & \\
                $0$ & $f(0)$ & $f'(0)$ & \\
                $1$ & $f(1)$ & $f(1) - f(0)$ & $f(1) - f(0) - f'(0)$
            \end{tabular}

            Therefore $p(x) = f(0) + f'(0) x + \p{f(1) - f(0) - f'(0)} x^2$.

        \item[(b)] % TODO: find error of integration formula
            Using the interpolation polynomial found in (a),
            $\dintt{0}{1}{f(x)}{x} \approx \dintt{0}{1}{p(x)}{x}$.
            \begin{align*}
                \dintt{0}{1}{p(x)}{x} &= \dintt{0}{1}{f(0) + f'(0) x + \p{f(1) - f(0) - f'(0)} x^2}{x} \\
                &= \eval{f(0) x + \frac{1}{2} f'(0) x^2 + \frac{1}{3}\p{f(1) - f(0) - f'(0)}x^3}{x=0}{1} \\
                &= f(0) + \frac{1}{2} f'(0) + \frac{1}{3}\p{f(1) - f(0) - f'(0)} \\
                &= \frac{2}{3} f(0) - \frac{1}{3} f(1) + \frac{1}{6} f'(0)
                \intertext{Therefore}
                \dintt{0}{1}{f(x)}{x} &\approx \frac{2}{3} f(0) - \frac{1}{3} f(1) + \frac{1}{6} f'(0)
            \end{align*}
            This can be expressed as the integration formula
            $\dintt{0}{1}{f(x)}{x} \approx a_0 f(0) + a_1 f(1) + b_0 f'(0)$, where
            $a_0 = \frac{2}{3}$, $a_1 = -\frac{1}{3}$, and $b_0 = \frac{1}{6}$.

        \item[(c)]
            \begin{tabular}{c|c|c|c}
                $x$ & $f(x)$ & & \\
                \midrule
                $c$ & $f(c)$ & & \\
                $c$ & $f(c)$ & $f'(c)$ & \\
                $c+h$ & $f(c+h)$ & $(f(c) - f(c+h))/h$ & $(f(c) - f(c+h))/h^2 - f'(c)/h$
            \end{tabular}
    \end{enumerate}

    \item % 4
    \item % 5
    \item % 6
    \begin{enumerate}
        \item[(a)]
            
        \item[(b)]
        \item[(c)]
    \end{enumerate}
\end{enumerate}
\end{document}
