\documentclass[11pt]{article}
\usepackage[letterpaper]{geometry}
\usepackage{MATH561}
\usepackage{SetTheory}
\usepackage{Derivative}
\allowdisplaybreaks

\begin{document}
\noindent \textbf{\Large{Caleb Logemann \\
MATH 561 Numerical Analysis I \\
Homework 4
}}

\begin{enumerate}
    \item[\#1]
        \begin{itemize}
            \item[(a)]
                Determine the principle error function of the general explicit
                two-stage Runge-Kutta method.

                The general explicit two-stage Runge-Kutta method can be
                described as follows.
                \begin{align*}
                    k_1 &= f(x,y) \\
                    k_2 &= f(x + \mu h, y + \mu h k_1) \\
                    \Phi(x, y; h) = \alpha_1 k_1 + \alpha_2 k_2
                \end{align*}

                To find the priniciple error function, first the local
                truncation error must be found.
                The local truncation error is defined as
                \[
                    T(x, y; h) = \Phi(x, y; h) - \frac{1}{h}\p{y(x + h) - y(x)}
                \]
                The principle error function is the functional coefficient of $h^p$
                in the local truncation error, when $p$ is the order of the method.
                Two-stage Runge-Kutta methods have in general an order of $p = 2$,
                so the principle error function is the coefficient of $h^2$.
                In order to find this the Taylor expansion of $\Phi(x, y; h)$ and
                $\frac{1}{h}\p{y(x + h) - y(x))}$ must be found, at least to the 
                $h^2$ term.

                First I will find the Taylor expansion of
                $\Phi(x, y; h) = \alpha_1 k_1 + \alpha_2 k_2$.
                The Taylor expansion of $k_1 = f(x,y)$ is just $f(x,y)$.
                The Taylor expansion of $k_2$ can be found as follows.
                \begin{align*}
                    k_2 &= f(x + \mu h, y + \mu h k_1) \\
                    &= f(x + \mu h, y + \mu h f(x,y)) \\
                    &= f(x,y) + f_x(x, y)(\mu h) + f_y(x,y)(\mu h f(x,y)) \\
                        &+ \frac{1}{2}\p{f_{xx}(x,y) (\mu h)^2 + 2f_{xy}(x,y)(\mu^2 h^2 f(x,y))
                        + f_{yy}(x,y)(\mu^2 h^2 f(x,y)^2} + O(h^3) \\
                    &= f(x,y) + \mu \p{f_x(x,y) + f(x,y) f_y(x,y)} h \\
                        &+\frac{1}{2} \mu^2 \p{f_{xx}(x,y) + 2 f(x,y) f_{xy}(x,y) + f(x,y)^2 f_{yy}(x,y)} h^2
                        + O(h^3) \\
                \end{align*}
                Now the Taylor expansion of $\Phi(x, y; h)$ can be expressed as follows.
                Note that moving forward all values or derivatives of $f$ will be
                evaluated at $(x,y)$.
                Thus $f = f(x,y)$, $f_x = f_x(x,y)$, $f_y = f_y(x,y)$, and so on.
                \begin{align*}
                    \Phi(x,y;h) &= \alpha_1 k_1 + \alpha_2 k_2 \\
                    &= \alpha_1 f + \alpha_2 \p{f + \mu \p{f_x + f f_y} h +
                        \frac{1}{2} \mu^2 \p{f_{xx} + f f_{xy} + f^2 f_{yy}} h^2 + O(h^3)} \\
                    &= (\alpha_1 + \alpha_2) f + \mu \alpha_2 \p{f_x + f f_y} h + 
                        \frac{1}{2} \alpha_2 \mu^2 \p{f_{xx} + f f_{xy} + f^2 f_{yy}} h^2 + O(h^3)
                \end{align*}

                Now that the Taylor expansion of $\Phi(x, y; h)$ has been found
                the Taylor expansion of $\frac{1}{h}\p{y(x + h) - y(x))}$ must be
                found and put in terms of $f$.
                \begin{align*}
                    
                \end{align*}
            \item[(b)]
            \item[(c)]
        \end{itemize}
\end{enumerate}
\end{document}
