\documentclass[11pt]{article}
\usepackage[letterpaper, margin = .75in]{geometry}
\usepackage{MATH561}
\usepackage{SetTheory}
\usepackage{Derivative}
\usepackage{Vector}
\usepackage{Complex}
\allowdisplaybreaks


\begin{document}
\noindent \textbf{\Large{Caleb Logemann \\
MATH 561 Numerical Analysis I \\
Homework 5
}}

\begin{enumerate}
    \item % #1
        \begin{enumerate}
            \item[(a)]
                Consider an explicit multistep method of the form
                \[
                    u_{n+2} + \alpha u_{n+1} - \alpha u_{n-1} - u_{n-2} =
                    h\p{\beta f_{n+1} + \gamma f_n + \beta f_{n-1}}
                \]
                Show that the parameters $\alpha, \beta, \gamma$ can be
                chosen uniquely so that the method has order $p = 6$.

                % Hint: to preserve symmetry and algebraic symmplicity define
                % the associated linear functional on the interval [-2, 2] instead
                % of [0, 4] as in Section 6.1.2

                In order to examine the order of the given multistep method, the
                algebraic degree of this method can be found using the linear
                operator shown below.
                \begin{align*}
                    Lu &= \sum{s = -2}{2}{\alpha_s u(s) - \beta_s u'(s)}
                    Lu &= -u(-2) - \alpha u(-1) + \alpha u(1) + u(2) - \beta u'(-1) - \gamma u'(0) - \beta u'(1)
                \end{align*}
                Since we are considering the algebraic degree and are working
                with polynomials, this linear operator is equivalent to the
                linear operator defined on $[0, 4]$.
                This is a result of the set of polynomials being closed under
                translation.
                
                In order for this method to have order $p = 6$, $Lu = 0$ for
                $u = 1, t, t^2, t^3, t^4, t^5, t^6$.
                This result in the following set of equations.
                \begin{align*}
                    0 &= -1 - \alpha + \alpha + 1 = 0 \\
                    0 &= 2 + \alpha + \alpha + 2 - \beta - \gamma - \beta = 4 + 2\alpha -2\beta - \gamma \\
                    0 &= -4 - \alpha + \alpha  + 4 + 2\beta  - 2\beta = 0 \\
                    0 &= 8 + \alpha + \alpha + 8 - 3\beta - 3\beta = 16 + 2\alpha - 3\beta \\
                    0 &= -16 - \alpha + \alpha + 16 + 4\beta - 4\beta = 0 \\
                    0 &= 32 + \alpha  + \alpha + 32 - 5\beta - 5\beta = 32 + 2\alpha - 10\beta \\
                    0 &= -64 - \alpha + \alpha + 64 + 6\beta - 6\beta = 0
                \end{align*}
                There are only three nontrivial equations.
                These are solved for the three variables below.
                \begin{align*}
                    0 &= 16 + 2\alpha - 3\beta \\
                    \beta &= \frac{16 + 2\alpha}{3} \\
                    0 &= 32 + 2\alpha - 10\beta \\
                    0 &= 32 + 2\alpha - 10\frac{16 + 2\alpha}{3}\\
                    0 &= 32 + 2\alpha - \frac{20}{3}\alpha - \frac{160}{3} \\
                    0 &= -\frac{64}{3} - \frac{14}{3}\alpha \\
                    \alpha &= -\frac{32}{7} \\
                    \beta &= \frac{16}{7} \\
                    0 &= 4 + 2\alpha -2\beta - \gamma \\
                    0 &= 4 + \frac{64}{7} - \frac{32}{7} - \gamma \\
                    \gamma &= \frac{244}{21}
                \end{align*}
                Therefore this method has order $p = 6$ if and only if $\alpha = -\frac{32}{7}$,
                $\beta = \frac{16}{7}$, and $\gamma = \frac{244}{21}$.
                This method can then be written as
                \[
                    u_{n+2} - \frac{32}{7} u_{n+1} + \frac{32}{7} u_{n-1} - u_{n-2} =
                    h\p{\frac{16}{7} f_{n+1} + \frac{244}{21}f_n + \frac{16}{7}f_{n-1}}
                \]

            \item[(b)]
                Discuss the stability properties of the method obtained in (a).

                % Outline: use the root condition, examine the roots of \alpha(t)

                The stability of the method found (a) can be determined by examining
                the characteristic polynomial and determining if it meets the root
                condition.

                The characteristic polynomial of this method is
                \[
                    \alpha(\xi) = \xi^4 - \frac{32}{7} \xi^3 + \frac{32}{7}\xi - 1
                \]
                The roots of this polynomial can be found by factoring
                \begin{align*}
                    \alpha(\xi) &= \xi^4 - \frac{32}{7} \xi^3 + \frac{32}{7}\xi - 1 \\
                    &= \frac{1}{7}\p{7\xi^4 - 32 \xi^3 + 32\xi - 7} \\
                    &= \frac{1}{7}\p{\xi + 1}\p{\xi - 1}\p{7\xi^2 - 32\xi + 7} \\
                \end{align*}
                The polynomial $7\xi^2 - 32\xi + 7$ has roots
                $\frac{32 \pm \sqrt{32^2 - 4\times7\times7}}{14} = \frac{16}{7} \pm \frac{6\sqrt{23}}{7}$
                The root $\frac{16}{7} + \frac{6\sqrt{23}}{7} > 1$, therefore $\alpha(\xi)$
                does not satisfy the root condition and this method is unstable.
        \end{enumerate}

    \item % #2 Done
        Construct a pair of four-step methods, one explicit, the other implicit,
        both have $\alpha(\xi) = \xi^4 - \xi^3$ and order $p = 4$,
        but global error constants that are equal in magnitude but opposite in
        sign.

        % See example in Gautschi page 437

        Since $\alpha(\xi) = \xi^4 - \xi^3$, this implies that $\alpha_4 = 1$
        and $\alpha_3 = -1$.
        Thus these four-step methods can also be expressed as
        \[
            u_{n+4} - u_{n+3} = h\p{\beta_4 f_{n+4} + \beta_3 f_{n+3} +
                \beta_2 f_{n+2} + \beta_1 f_{n+1} + \beta_0 f_n}
        \]
        For the explicit method $\beta_4 = 0$ and for the implicit method
        $\beta_4 \neq 0$.

        In order for these four step methods to have order $p = 4$, the function
        $\delta(\xi) = \frac{\alpha(\xi)}{\ln{\xi}} - \beta(\xi)$ must have
        a zero at $\xi = 1$ of multiplicity $4$.

        First consider the Taylor expansion of $\frac{\alpha(\xi)}{\ln{\xi}}$.
        \begin{align*}
            \ln{\xi} &= \ln{1 + \xi - 1} \\
            &= (\xi - 1) - \frac{1}{2}(\xi - 1)^2 + \frac{1}{3}(\xi - 1)^3
                - \frac{1}{4}(\xi - 1)^4 + \cdots \\
            \frac{\alpha(\xi)}{\ln{\xi}} &= \frac{\xi^3(\xi - 1)}{(\xi - 1) - \frac{1}{2}(\xi - 1)^2 + \frac{1}{3}(\xi - 1)^3
                - \frac{1}{4}(\xi - 1)^4 + \cdots} \\
            &= 1 + \frac{7}{2}(\xi - 1) + \frac{53}{12}(\xi - 1)^2 + \frac{55}{24}(\xi - 1)^3 + \frac{251}{720}(\xi - 1)^4 + \cdots
        \end{align*}

        For $\delta(\xi)$ to have a root of order $4$ at $\xi = 1$, the Taylor
        expansion of $\delta(\xi)$ about $\xi=1$ must have $4$ as its least power.
        Therefore $\beta(\xi)$ must equal the first four terms in the Taylor
        expansion of $\alpha(\xi)/\ln(\xi)$, that is
        \begin{align*}
            \beta(\xi) &= 1 + \frac{7}{2}(\xi - 1) + \frac{53}{12}(\xi - 1)^2 + \frac{55}{24}(\xi - 1)^3 \\
            &= 1 - \frac{7}{2} + \frac{7}{2}\xi + \frac{53}{12}(\xi^2 - 2\xi + 1) + \frac{55}{24}(\xi^3 - 3\xi^2 + 3\xi - 1) \\
            &= 1 - \frac{7}{2} + \frac{53}{12} - \frac{55}{24} + \p{\frac{7}{2} - \frac{53}{6} + \frac{55}{8}}\xi + \p{\frac{53}{12} - \frac{55}{8}}\xi^2 + \frac{55}{24}\xi^3 \\
            &= -\frac{3}{8} + \frac{37}{24}\xi - \frac{59}{24}\xi^2 + \frac{55}{24}\xi^3 \\
        \end{align*}
        Using this polynomial for $\beta(\xi)$ results in an explicit method
        because $\beta_4 = 0$.
        The global error constant is the coefficient of $(\xi - 1)^4$ divided
        by the constant coefficient, so in this case the global error constant is $\frac{251}{720}$.
        Thus we have created the explicit four step method shown below with order $p = 4$.
        \[
            u_{n+4} - u_{n+3} = h\p{\frac{55}{24}f_{n+3} +
                -\frac{59}{24}f_{n+2} + \frac{37}{24}f_{n+1} - \frac{3}{8}f_n}
        \]

        To create an implicit four-step method, the following polynomial must
        be used to describe $\beta$ instead
        \[
            \beta(\xi) = 1 + \frac{7}{2}(\xi - 1) + \frac{53}{12}(\xi - 1)^2 + \frac{55}{24}(\xi - 1)^3 + b(\xi - 1)^4 \\
        \]
        where $b \neq 0$ and $b \neq \frac{251}{720}$.
        If $b = \frac{251}{720}$, then the method would have order $p = 5$.
        In this case, $\delta(\xi) = \p{\frac{251}{720} - b}(\xi - 1)^4 + \cdots$.
        Therefore the global error constant for this method is $\frac{251}{720} - b$.
        In order for this global error constant to be equal in modulus and
        opposite in sign to the previous global error constant
        $\frac{251}{720} - b = -\frac{251}{720}$.
        This implies that $b = \frac{251}{360}$.

        Therefore the $\beta$-polynomial can be simplied as follows
        \begin{align*}
            \beta(\xi) &= 1 + \frac{7}{2}(\xi - 1) + \frac{53}{12}(\xi - 1)^2 + \frac{55}{24}(\xi - 1)^3 + \frac{251}{360}(\xi - 1)^4 \\
            &= -\frac{3}{8} + \frac{37}{24}\xi - \frac{59}{24}\xi^2 + \frac{55}{24}\xi^3 + \frac{251}{360}(\xi - 1)^4 \\
            &= -\frac{3}{8} + \frac{37}{24}\xi - \frac{59}{24}\xi^2 + \frac{55}{24}\xi^3 + \frac{251}{360}(\xi^4 -4 \xi^3 + 6 \xi^2 - 4\xi + 1) \\
            &= \frac{29}{90} - \frac{449}{360} \xi + \frac{69}{40} \xi^2 - \frac{179}{360} \xi^3 + \frac{251}{360} \xi^4
        \end{align*}
        This results in the following implicit four-step method of order $p=4$.
        \[
            u_{n+4} - u_{n+3} = h\p{\frac{251}{360}f_{n+4} - \frac{179}{360}f_{n+3} +
                \frac{69}{40}f_{n+2} - \frac{448}{360}f_{n+1} + \frac{29}{90}f_n}
        \]

    \item % #3
        Consider the model problem
        \[
            \d{y}{x} = -\omega\p{y - a(x)}, 0 \le x \le 1, y(0) = y_0
        \]
        where $\omega > 0$ and (i) $a(x) = x^2$, $y_0 = 0$; and (ii) $a(x) = e^x$, $y_0 = 1$
        \begin{enumerate}
            \item[(a)] % Done
                In each of the cases (i) and (ii), solve the differential
                equation exactly.

                For case (i) this differential equation can be rewritten as
                \[
                    \d{y}{x} = -\omega\p{y - x^2}
                \]
                for $0 \le x \le 1$ and $y(0) = 0$.
                This is equivalent to 
                \[
                    \omega\p{y - x^2} + \d{y}{x} = 0
                \]
                This is not an exact differential equation but it can be made
                into an exact differential equation by multiplying by an
                integrating factor, $\mu(x)$.

                \begin{align*}
                    \mu(x)\omega\p{y - x^2} + \mu(x)\d{y}{x} &= 0 \\
                    \pd{\mu(x)\omega\p{y - x^2}}{y} &= \mu(x) \omega \\
                    \pd{\mu(x)}{x} &= \mu'(x) \\
                    \mu(x) \omega &= \d{\mu}{x} \\
                    \mu(x) &= e^{\omega x}
                    \intertext{This results in the equivalent exact differential equation}
                    e^{\omega x} \omega \p{y - x^2} + e^{\omega x} \d{y}{x} &= 0
                    \intertext{To solve this type of equation a function
                        $\Psi(x, y)$ must be found such that}
                    \pd{\Psi(x, y)}{x} &= e^{\omega x} \omega \p{y - x^2}
                    \intertext{and}
                    \pd{\Psi(x, y)}{y} &= e^{\omega x}
                    \intertext{Solving for $\Psi$}
                    \Psi(x,y) &= \intt{e^{\omega x} \omega \p{y - x^2}}{x} \\
                              &= \intt{\omega y e^{\omega x}}{x} - \intt{\omega e^{\omega x} x^2}{x} + h(y) \\
                              &= y e^{\omega x} - e^{\omega x}\p{\frac{\omega^2 x^2 - 2 \omega x + 2}{\omega^2}} + h(y) \\
                    \pd{\Psi(x,y)}{y} &= e^{\omega x} + h'(y) \\
                    e^{\omega x} + h'(y) &= e^{\omega x} \\
                    h(y) &= C \\
                    \Psi(x,y) &= y e^{\omega x} - e^{\omega x}\p{\frac{\omega^2 x^2 - 2 \omega x + 2}{\omega^2}} + C \\
                \end{align*}
                Now this exact differential equation is equivalent to the equation
                \begin{align*}
                    y e^{\omega x} - e^{\omega x}\p{\frac{\omega^2 x^2 - 2 \omega x + 2}{\omega^2}} &= C \\
                    y e^{\omega x} &= e^{\omega x}\p{\frac{\omega^2 x^2 - 2 \omega x + 2}{\omega^2}} + C \\
                    y(x) &= \frac{\omega^2 x^2 - 2 \omega x + 2}{\omega^2} + e^{-\omega x}C \\
                    y(x) &= x^2 - \frac{2}{\omega} x + \frac{2}{\omega^2} + e^{-\omega x}C \\
                \end{align*}

                Now the initial conditions must be used to solve for $C$.
                \begin{align*}
                    y(0) &= \frac{2}{\omega^2} + C \\
                    0 &= \frac{2}{\omega^2} + C \\
                    C &= -\frac{2}{\omega^2}
                \end{align*}

                Therefore the exact solution to (i) is
                \[
                    y(x) = x^2 - \frac{2}{\omega} x + \frac{2}{\omega^2}\p{1 - e^{-\omega x}} \\
                \]

                For case (ii) this differential equation can be rewritten as
                \[
                    \d{y}{x} = -\omega\p{y - e^x}
                \]
                for $0 \le x \le 1$ and $y(0) = 1$.

                The same integrating factor as in part (i) can be used
                \begin{align*}
                    \omega e^{\omega x}\p{y - e^x} + e^{\omega x}\d{y}{x} &= 0
                    \intertext{To solve this type of equation a function
                        $\Psi(x, y)$ must be found such that}
                    \pd{\Psi(x, y)}{x} &= e^{\omega x} \omega \p{y - e^x}
                    \intertext{and}
                    \pd{\Psi(x, y)}{y} &= e^{\omega x}
                    \intertext{Solving for $\Psi$}
                    \Psi(x, y) &= \intt{e^{\omega x}}{y} \\
                               &= y e^{\omega x} + h(x) \\
                    \pd{\Psi(x,y)}{x} &= \omega y e^{\omega x} + h'(x) \\
                    \omega y e^{\omega x} + h'(x) &= e^{\omega x} \omega \p{y - e^x} \\
                    \omega y e^{\omega x} + h'(x) &= e^{\omega x} \omega y - e^{\omega x} \omega e^x \\
                    h'(x) &=  - \omega e^{(\omega + 1) x} \\
                    h(x) &= - \frac{\omega}{\omega + 1} e^{(\omega + 1) x} + C \\
                    \Psi(x, y) &= y e^{\omega x} - \frac{\omega}{\omega + 1} e^{(\omega + 1) x} + C \\
                \end{align*}
                Now this exact differential equation is equivalent to the equation
                \begin{align*}
                    y e^{\omega x} - \frac{\omega}{\omega + 1} e^{(\omega + 1) x} &= C \\
                    y e^{\omega x} &= \frac{\omega}{\omega + 1} e^{(\omega + 1) x} + C \\
                    y(x) &= e^{-\omega x}\p{\frac{\omega}{\omega + 1} e^{(\omega + 1) x} + C} \\
                    y(x) &= \frac{\omega}{\omega + 1} e^{x} + Ce^{-\omega x} \\
                \end{align*}
                Then the initial conditions can be used to solve for $C$.
                \begin{align*}
                    y(0) &= \frac{\omega}{\omega + 1} + C \\
                    1 &= \frac{\omega}{\omega + 1} + C \\
                    C &= 1 - \frac{\omega}{\omega + 1} \\
                    C &= \frac{1}{\omega + 1}
                \end{align*}

                Thus the exact solution to this differential equation is
                \begin{align*}
                    y(x) &= \frac{\omega}{\omega + 1} e^{x} + \frac{1}{\omega + 1}e^{-\omega x} \\
                    y(x) &= \frac{\omega e^x + e^{-\omega x}}{\omega + 1}\\
                \end{align*}

            \item[(b)]
                In each of the cases (i) and (ii), apply the kth-order Adams-Bashford method and
                kth-order Adams predictor/corrector method, for $k = 4$, using exact starting values and
                step lengths h = 1/20 , 1/40 , 1/80 , 1/160. 
                Plot the exact values $y_n$ and numerical solution $u_n$, and 
                check the accuracy of the methods.
                Try $\omega = 1, 10, 50$.
                Summarize your results.
        \end{enumerate}
\end{enumerate}
\end{document}
