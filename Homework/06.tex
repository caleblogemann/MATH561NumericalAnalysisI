\documentclass[11pt]{article}
\usepackage[letterpaper, margin = .75in]{geometry}
\usepackage{MATH561}
\usepackage{SetTheory}
\usepackage{Derivative}
\usepackage{Vector}
\usepackage{Complex}
\allowdisplaybreaks


\begin{document}
\noindent \textbf{\Large{Caleb Logemann \\
MATH 561 Numerical Analysis I \\
Final Assignment
}}

\begin{enumerate}
    \item % #1 Done
        Let $x_1, x_2, \ldots, x_n$, for $n > 1$, be machine numbers.
        Their product can be computed by the alogirithm
        \begin{align*}
            p_1 &= x_1 \\
            p_k &= fl(x_k p_{k-1}), k = 2, 3, \ldots, n
        \end{align*}
        \begin{enumerate}
            \item[(a)]
                Find an upper bound for the relative error in terms of the
                machine precision $eps$ and $n$.

                The relative error is given by
                \[
                    \frac{p_n - x_1x_2\cdots x_n}{x_1x_2\cdots x_n}
                \]

                First consider $p_k$.
                \begin{align*}
                    p_k &= fl(x_k p_{k-1}) \\
                        &= x_k p_{k-1} (1 + \epsilon_k)
                    \intertext{Where $\abs{\epsilon_k} < eps$, for $k = 1, \cdots, n$}
                    &< x_k p_{k-1} (1 + eps)
                    \intertext{Applying this recursively to $p_n$, we see that}
                    p_n &< x_n p_{n-1} (1 + eps) \\
                        &< x_n x_{n-1} p_{n-2} (1 + eps)^2 \\
                        &< x_n x_{n-1} x_{n-2} p_{n-3} (1 + eps)^3 \\
                        &\vdots \\
                        &< x_n x_{n-1} \cdots x_1 (1 + eps)^{n-1}
                \end{align*}

                Therefore the relative error can be bounded as follows
                \begin{align*}
                    E &= \abs{\frac{p_n - x_1x_2\cdots x_n}{x_1x_2\cdots x_n}} \\
                    &< \abs{\frac{x_n x_{n-1} \cdots x_1 (1 + eps)^{n-1} - x_1x_2\cdots x_n}{x_1x_2\cdots x_n}} \\
                    &= \abs{\frac{x_1x_2\cdots x_n\p{(1 + eps)^{n-1} - 1}}{x_1x_2\cdots x_n}} \\
                    &= (1 + eps)^{n-1} - 1 \\
                \end{align*}
                Therefore the upper bound for the relative error is
                $E < (1 + eps)^{n-1} - 1$.

            \item[(b)]
                For any integer $r$ that satisfies $r \times eps < \frac{1}{10}$,
                show that
                \[
                    (1 + eps)^r - 1 < 1.06 \times r \times eps
                \]
                Hence for $n$ not too large, simplify the answer given in (a).

                % Use the Binomial Theorem
                Using the Binomial Thereom, $(1 + eps)^r$ can be expanded.
                \begin{align*}
                    (1 + eps)^r - 1 &= \sum{i = 0}{r}{\binom{r}{i} 1^{r - i} eps^i} - 1 \\
                    &= \sum{i = 1}{r}{\binom{r}{i} eps^i} \\
                    &= r \cdot eps + \binom{r}{2} eps^2 + \binom{r}{3} eps^3 + \cdots + eps^r \\
                    &= r \cdot eps + \frac{r(r-1)}{2} eps^2 + \frac{r(r-1)(r-2)}{3!} eps^3 + \cdots + eps^r \\
                    &= r \cdot eps \p{1 + \frac{r-1}{2} eps + \frac{(r-1)(r-2)}{3!} eps^2 + \cdots + \frac{(r-1)(r-2)\cdots(1)}{r!} eps^{r-1}} \\
                    \intertext{Since $r \times eps < \frac{1}{10}$, $(r-i) eps < \frac{1}{10}$ for any $0 < i < r$}
                    &< r \cdot eps \p{1 + \frac{1}{2} \frac{1}{10} + \frac{1}{3!} \p{\frac{1}{10}}^2 + \cdots + \frac{1}{r!} \p{\frac{1}{10}}^{r-1}} \\
                    &= r \cdot eps \sum{k=0}{r-1}{\frac{1}{k!} \p{\frac{1}{10}}^{k-1}} \\
                    &= r \cdot eps \cdot 10 \sum{k=1}{r-1}{\frac{1}{k!} \p{\frac{1}{10}}^{k}}
                    \intertext{This expression is certainly less than extending
                        the sum to infinity because all of the terms are postive.
                        Also this sum is the Taylor series for $e^x - 1$.}
                    &< r \cdot eps \cdot 10 \sum{k=1}{\infty}{\frac{1}{k!} \p{\frac{1}{10}}^{k}} \\
                    &= r \cdot eps \cdot 10 \p{e^{1/10} - 1} \\
                    &\approx 1.05171 r \cdot eps \\
                    &< 1.06 r \cdot eps
                \end{align*}
                This result can now be used to simplify the result of part (a).
                Now if $n$ is not too large, then $\abs{E} < 1.06 (n-1)eps$.
        \end{enumerate}

    \item % #2

    \item % #3

    \item % #4
        Let $a = x_0 < x_1 < \cdots < x_n = b$ be a partition of $[a, b]$.
        Consider a function $f \in C^{\infty}[a, b]$.
        \begin{enumerate}
            \item[(a)]
                Define what it means for a function $S$ to be a linear spline
                that interpolates $f$ at all the points $x_i$ for
                $i = 0, 1, \ldots, n$.
                Give a formula for $S$ in terms of the point values of $f$.

                In order to define the linear spline, I will first define a set of
                linear basis functions.
                Let $B_i$ for $i = 1, 2, \ldots, n-1$ be defined on $[a, b]$ as follows.
                \begin{align*}
                    B_i(x) = 
                    \begin{cases}
                        \frac{x - x_{i-1}}{x_i - x_{i-1}} & x_{i-1} \le x \le x_i \\
                        \frac{x_{i+1} - x}{x_{i+1} - x_i} & x_i < x \le x_{i+1} \\
                        0 & \text{otherwise}
                    \end{cases}
                \end{align*}
                Also let $B_1$ and $B_n$ be defined as follows
                \begin{align*}
                    B_1(x) &=
                    \begin{cases}
                        \frac{x - x_{n-1}}{x_n - x_{n-1}}& a = x_{0} \le x \le x_1 \\
                        0 & x > x_1
                    \end{cases} \\
                    B_n(x) &=
                    \begin{cases}
                        \frac{x_1 - x}{x_1 - x_0}  & x_{n-1} \le x \le x_n = b \\
                        0 & x < x_{n-1}
                    \end{cases}
                \end{align*}

                A linear spline on $[a, b]$ that interpolates $f$ on the
                partition $\set{x_i}_{i=0}^{n}$ is a function $S(x)$ that is a
                linear combination of the basis functions $B_i$ such that
                $S(x_i) = f(x_i)$ for $i = 0, 1, \ldots, n$.

                Thus a formula for $S(x)$ could be written as
                $S(x) = \sum{i=0}{n}{f(x_i)B_i(x)}$.

            \item[(b)]
                Let $h = \max*_{0 \le i \le n-1}(x_{i+1} - x_i)$.
                Derive an upper bound on $\abs{f(x) - S(x)}$ on $x \in [a, b]$.
                Use this to prove that $\lim{h \to 0}{\abs{f(x) - S(x)}} = 0$
                for $x \in [a, b]$ and state the rate of convergence.

            \item[(c)]
                Define what it means for $S$ to be a clamped cubic spline that
                interpolates $f$ at all the points $x_i$, for $i = 0, 1, \ldots, n$.

        \end{enumerate}

    \item % #5
\end{enumerate}
\end{document}
